% Copyright (C) 2007 Technical University of Liberec.  All rights reserved.
%
% Please make a following reference to Flow123d on your project site if you use the program for any purpose,
% especially for academic research:
% Flow123d, Research Centre: Advanced Remedial Technologies, Technical University of Liberec, Czech Republic
%
% This program is free software; you can redistribute it and/or modify it under the terms
% of the GNU General Public License version 3 as published by the Free Software Foundation.
%
% This program is distributed in the hope that it will be useful, but WITHOUT ANY WARRANTY;
% without even the implied warranty of MERCHANTABILITY or FITNESS FOR A PARTICULAR PURPOSE.
% See the GNU General Public License for more details.
%
% You should have received a copy of the GNU General Public License along with this program; if not,
% write to the Free Software Foundation, Inc., 59 Temple Place - Suite 330, Boston, MA 021110-1307, USA.
%
%%%%%%%%%%%%%%%%%%%%%%%%%%%%%%%%%%%%%%%%%%%%%%%%%%%%%%%%%%%%%%%%%%
%
% use PDFLatex to compile this
%

\documentclass[a4paper]{article}

% our own flow_doc.sty
%\usepackage{flow_doc}

%\usepackage{rotating}
%\usepackage{pdflscape}

\usepackage{amssymb, amsmath, amsthm}
\newtheorem{theorem}{Theorem}


\usepackage{array}
\usepackage{longtable}
\usepackage[usenames,dvipsnames]{color}   %colors
%\usepackage{colortbl}   %colorful tables
\usepackage{tabularx}
\usepackage{graphicx} %[dvips]
% it is note used \usepackage{cooltooltips}

%these two can be found in caption package
%\usepackage{caption}
%\usepackage{subcaption}

\usepackage[numbers]{natbib}

%\usepackage{fancyvrb}   % extended verbatim environments (for examples of IO files)

%\usepackage{multicol}
\usepackage{etoolbox}


%%%%%%%%%%%%%%%%%%%%%%%%%%%%%%%%%%%%%%%%%%%%%%%%%%%%%%%%%%%%%%%%%%%%%%%%%%%%
% macro for units 
\def\UNIT#1#2{\ifstrempty{#2}{}{%
\ifstrequal{#2}{1}{\mathrm{#1}}{\mathrm{#1}^{#2}}%
}}
\def\units#1#2#3{\ifstrempty{#1#2#3}{$[-]$}{$[ \UNIT{kg}{#1}\UNIT{m}{#2}\UNIT{s}{#3} ]$}}       %with brackets
\def\unitss#1#2#3{\ifstrempty{#1#2#3}{$-$}{$ \UNIT{kg}{#1}\UNIT{m}{#2}\UNIT{s}{#3} $}}  %without brackets


\newcommand{\vari}[1]{{\it #1}}
\newcommand{\ditem}[2]{\item[\vari{#1} {\tt #2}]}
\newenvironment{fileformat}{\tt\begin{flushleft}}{\end{flushleft}}

%%%%%%%%%%%%%%%%%%%% specific math macros
\def\prtl{\partial}
\def\vc#1{\mathbf{\boldsymbol{#1}}}     % vector
\def\tn#1{{\mathbb{#1}}}    % tensor
\def\abs#1{\lvert#1\rvert}
\def\Abs#1{\bigl\lvert#1\bigr\rvert}
\def\div{{\rm div}}
\def\Lapl{\Delta}
\def\grad{\nabla}
\def\Real{{\mathbf R}}
\def\d {\,{\rm d}}
\def\Natural{\mathbf N}
\def\norm#1{\|#1\|}
\def\yy{{\vc y}}

\newcommand{\note}[2]{{\color{blue} \textbf{ #1:} \textit{#2}}}
%% ini_table members
%%%%%%%%%%%%%%%%%%%% specific math macros


%%%%%%%%%%%%%%%%%%%%%%%%%%%%%%%%%%%%%%%%%%%%%%%%%%%%%%%%%%%%%%%%%%%%%%%%%%%%%%%%%%%%%%%%%%%%% BEGIN DOCUMENT
%% set specific page layout
%\addtolength{\textwidth}{2cm}
%\addtolength{\hoffset}{-1.5cm}
%\addtolength{\textheight}{4cm}
%\addtolength{\voffset}{-2.5cm}
\begin{document}

\section{Introduction}
\note{JB}{
\begin{itemize}
 \item Motivation, background (granite, impact of small scale to large scale)
 \item 2d domain, notation (figure with boundary perpendicular to fracture)
 \item Darcy flow, Solute transport equations, basic notation
\end{itemize}
}

Deep subsurface deposits in plutonic rock represents one of possible solution for final storage of nuclear waste. The primary reason is 
small hydraulic permeability of the bulk rock and thus slow migration of a possible leakage due to ground water flow.  On the other hand, 
granitoid formations contain fractures that may form a network of preferential paths with low volumetric water flow rate but with 
high velocity. The preferential paths pose a risk of fast transport of small amount of contaminant but in potentially dangerous concentrations.
The large scale effect of the small scale fractures is challenging for numerical simulations since direct discretization requires highly refined
computational mesh. One possible solution is to model fractures as lower dimensional objects and introduce their coupling with surrounding continuum.
\note{JB}{ Overview of other papers, namely: Martin et al \cite{martin_modeling_2005} (Darcy, classical derivation, numeric), Angot et al \cite{angot_asymptotic_2009} 
(Darcy, curved fracture, definition of trace operator for immersed fracture, finite volume scheme, convergence, our case not covered); Fumagalli and Scotti (transport, classical derivation).}


\note{JS}{
MODELLING FRACTURES AS INTERFACES - survey of works and results, lack of rigorous justification (either formal derivation or error estimates just for particular approximation scheme).
}

\note{JS}{
Only basic notation in introduction (2d domain, Darcy, AD eq.) remaining in derivation of abstract fracture model.
}

\begin{figure}[h]
\centering
\includegraphics[width=12cm]{figures/full_model_domain}
\label{fig:omegas}
\caption{The domain of the full model (left) and the reduced geometry (right).}
\end{figure}


We consider a bounded domain $\Omega \subset \Real^d$, $d=2,3$ with a Lipschitz boundary, see Figure \ref{fig:omegas} left). The domain $\Omega$ contains 
a fracture $\Omega_f:=\Omega\cap \big(\Real^{d-1}\times(-\delta/2,\delta/2)\big)$ 
with the aperture $\delta>0$ surrounded by the matrix domain $\Omega_m:=\Omega\setminus\overline\Omega_f$.  The fracture interacts with the matrix domain on the interfaces
$\gamma_1:=\Omega\cap\big( \Real^{d-1}\times \{-\delta/2\}\big)$ and $\gamma_2:=\Omega\cap \big( \Real^{d-1}\times \{\delta/2\}\big)$. Further, we introduce reduced geometry (see Figure \ref{fig:omegas})
where the fracture domain is represented by the interface $\gamma:=\Omega\cap\big(\Real^{d-1}\times\{0\}\big)$ in its center. 

We are interested in a steady Darcy flow on domain $\Omega$:
\begin{subequations}
\label{eq:darcy_flow}
\begin{align}
    \div \vc{q_h} = f_h \mbox{ in }\Omega, \\
    \vc q_h = -\tn K \grad h, \\
    h=0\mbox{ on }\partial\Omega
\end{align}
\end{subequations}
where $\vc q$ is the Darcy flux, $f_h$ is the source density, $\tn K$ is the hydraulic conductivity tensor, and $h$ is the piezometric head.
Further, we consider also related solute transport equation:
\begin{align}
    \label{eq:solute_transport}
    \prtl_t (\theta c) + \div(q_c) = f_c \mbox{ in }\Omega,\\
    \vc q_c = \theta c \vc q_h - \tn D \grad c, \\
    c=0\mbox{ on }\partial\Omega
\end{align}
where $\theta$ is the porosity, $c$ is the solute concentration, $q_c$ is the total solute flux, and $\tn D$ is the diffusion-dispersion tensor that in general depends on $q_h$.

Both transport processes are particular cases of an abstract advection-diffusion problem which admits an equivalent formulation
emphasizing the structure of the domain $\Omega$:
\begin{align}
  \label{eq:fr:continuity}
  \prtl_t w_i + \div \vc j_i &= f_i&&  \text{in } \Omega_i,\ i=m,f,\\
  \label{eq:fr:flux}
  \vc j_i &= - \tn A_i\grad u_i + \vc b_i w_i&& \text{in } \Omega_i,\ i=m,f,\\
  \label{eq:fr:Dirichlet}
  u_i &= u_f&& \text{on } \gamma_i,\ i=1,2,\\
  \label{eq:fr:Neumann}
  \vc j_i \cdot \vc n &= \vc j_f \cdot \vc n&& \text{on } \gamma_i,\ i=1,2,
\end{align}
where $w_i=w_i(u_i)$ is the conservative quantity and $u_i$ is the principal unknown, $\vc j_i$ is the flux of $w_i$, $f_i$ is the source term,
$\tn A_i$ is the diffusivity tensor and $\vc b_i$ is the velocity field. 

In this paper we shall study the relation of the above problem to the so-called asymptotic problem on the reduced geometry.
\note{JB}{Proposed names:
\begin{itemize}
 \item continuum-fracture model
 \item reduced fracture model
 \item averaged fracture model
\end{itemize}
}

The aim of this paper is to justify the latter problem as an approximation of the first one in the case when the cross section $\delta$ is small.
In particular, we shall prove that
\[ \bar u - u_f \approx \delta\quad\mbox{and}\quad u_{|\Omega_m}-u_m \approx \delta^{3/2} \]
in a suitable sense.


The organization of the paper is as follows.
In the next section we present the formal derivation of the asymptotic model.
The main theoretical result on the error analysis is stated and proved in section \ref{sc:error_estimate}.
Finally, in section \ref{sc:numerics} we show numerical results which confirm the error estimates.











\section{Asymptotic model for abstract advection-diffusion problem}
\label{sc:ad_on_fractures}

In this section, we shall derive a model for the abstract advection-diffusion process $(\ref{eq:fr:continuity}--\ref{eq:fr:Neumann}$
on the reduced geometry inspired by the paper \citet{martin_modeling_2005}.


By $x$, $\vc y$ we denote the normal and the tangential coordinate of a point in $\Omega_f$.
Accordingly, $\partial_x$, $\nabla_{\vc y}:=(0,\partial_{y_1},\ldots,\partial_{y_{d-1}})^\top$ stands for the crosswind and the tangential derivative, respectively.
The symbol $\bar w:=\frac1\delta\int_{-\delta/2}^{\delta/2} w(x,\cdot)~dx$ will denote the integral mean of a function $w$ across the fracture opening.


We assume that the diffusive tensor $\tn A_f$ is symmetric positive definite 
with one eigenvector in the direction $\vc n$. Consequently the tensor has the form:
\[
 \tn A_f = \begin{pmatrix} 
            a_n & 0  \\
            0 & \tn A_t
       \end{pmatrix}
\]
Furthermore, we assume that $\tn A_f(x, \vc y)=\tn A_f(\vc y)$ is constant in the normal direction.







Our next aim is to integrate equations on the fracture $\Omega_f$ in the normal direction 
and obtain their approximations on the surface $\gamma$ running through the middle of the fracture. 
For the sake of clarity, we will not write subscript $f$ for quantities on the fracture. 
To make the following procedure mathematically correct we have to assume that the functions
$\prtl_x w$, $\prtl_x \grad_{\vc y} u$, $\prtl_x \vc b_{\vc y}$ are continuous and bounded in $\Omega_f$. Here and later on, 
$\vc b_x=(\vc b \cdot \vc n)\, \vc n$ is the normal part of the velocity field and $\vc b_{\vc y} = \vc b - \vc b_x$ is the tangential part.
The same notation will be used for normal and tangential part of the field $\vc j$.

We integrate \eqref{eq:fr:continuity} over the fracture opening $[-\delta/2,\delta/2]$ and use approximations to get
\begin{equation}
    \label{eq:fracture_continuity}
   \prtl_t (\delta W) - \vc j_2 \cdot \vc n_2 - \vc j_1 \cdot \vc n_1 + \div \vc J = \delta F,
\end{equation}
where for the first term, we have used mean value theorem, first order Taylor expansion, 
and boundedness of $\prtl_x w$ to obtain approximation:
\[
    \int_{-\frac\delta2}^{\frac\delta2} w(x,\vc y) \d x=\delta w(\xi_{\vc y}, \vc y) = \delta W(\vc y) + O(\delta^2\abs{\prtl_x w}),
\]
where
\[
    W(\vc y):=w(0,\vc y)=w(u(0,\vc y))=:w(U(\vc y)).
\]
Next two terms in \eqref{eq:fracture_continuity} come from the exact integration 
of the divergence of the normal flux $\vc j_x$.
Integration of the divergence of the tangential flux $\vc j_{\vc y}$ gives the fourth term, where we introduced
\[
\vc J(\vc y) := \int_{-\frac\delta2}^{\frac\delta2} \vc j_{\vc y}(x, \vc y) \d x.
\]
In fact, this flux on $\gamma$ is scalar for the case $d=2$. Finally, we integrate the right-hand side to get 
\[
    \int_{-\frac\delta2}^{\frac\delta2} f(x,\vc y) \d x = \delta F(\vc y) + O(\delta^2\abs{\prtl_x f}),\quad F(\vc y):=f(0,\vc y). 
\]


Due to the particular form of the tensor $\tn A_f$, we can separately integrate the tangential and the normal
part of the flux given by \eqref{eq:fr:flux}. Integrating the tangential part and using approximations
\[
    \int_{-\frac\delta2}^{\frac\delta2}  \grad_{\vc y} u(x, \vc y) \d x = \delta \grad_{\vc y} u (\xi_{\vc y}, \vc y) 
    = \delta \grad_{\vc y} U( \vc y) + O\big( \delta^2 \abs{\prtl_x\grad_{\vc y} u} \big) 
\]
and
\[
 \int_{-\frac\delta2}^{\frac\delta2} \big(\vc b_{\vc y} w\big)(x, \vc y) \d x 
  = \delta \vc B(\vc y) W(\vc y) + O\big(\delta^2 \abs{\prtl_x(\vc b_{\vc y} w)} \big)
\]
where
\[
  \vc B(\vc y) := \vc b_{\vc y}(0, \vc y),
\]
we obtain
\begin{equation}
    \label{eq:fracture_darcy}
   \vc J = -\tn A_t \delta \grad_{\vc y} U + \delta \vc B W + O\big(\delta^2(\abs{\prtl_x\grad_{\vc y} u}+\abs{\prtl_x(\vc b_{\vc y} w)})\big).
\end{equation}


So far, we have derived equations for the state quantities $U$ and $\vc J$ in the fracture manifold $\gamma$. In order to
get a well possed problem, we have to prescribe two conditions for the boundaries $\gamma_i$, $i=1,2$. To this end, we
perform integration of the normal flux $\vc j_x$, given by \eqref{eq:fr:flux}, separately for the left and right half of the fracture.
Similarly as before we use approximations
\[
 \int_{-\delta/2}^0 \vc j_x \d x = (\vc j_1 \cdot \vc n_1)\frac{\delta}{2} + O(\delta^2 \abs{\prtl_x \vc j_x})
\]
and 
\[
 \int_{-\delta/2}^0 \vc b_x w \d x = (\vc b_1 \cdot \vc n_1)\tilde{w}_1\frac{\delta}{2} + O(\delta^2 \abs{\prtl_x \vc b_x}\abs{w} + \delta^2\abs{\vc b_x}\abs{\prtl_x w})
\]
and their counterparts on the interval $(0,\delta/2)$ to get
\begin{align}
    \label{eq:fracture_normal_1}
     \vc j_1 \cdot \vc n_1 &= -\frac{2a_n}{\delta} (U - u_1) + \vc b_1\cdot \vc n_1 \tilde{w}_1\\
    \label{eq:fracture_normal_2}
    \vc j_2 \cdot \vc n_2 &= -\frac{2a_n}{\delta} (U - u_2) + \vc b_2\cdot \vc n_2 \tilde{w}_2
\end{align}
where $\tilde w_i$ can be any convex combination of $w_i$ and $W$. Equations \eqref{eq:fracture_normal_1}  
and \eqref{eq:fracture_normal_2} have meaning of a semi-discretized flux from domains $\Omega_i$ into fracture.
In order to get a stable numerical scheme, we introduce a kind of upwind already on this level using a different convex 
combination for each flow direction:
\begin{align}
   \notag 
   \vc j_i \cdot \vc n_i
       = &-\sigma_i (U - u_i)      \\ 
   \notag
      &+ \big[\vc b_i\cdot \vc n_i\big]^{+} \big(\xi w_i + (1-\xi) W\big)       \\
      \label{eq:fracture_normal}
      &+ \big[\vc b_i\cdot \vc n_i\big]^{-} \big((1-\xi) w_i + \xi W\big), \qquad i=1,2
\end{align}
where $\sigma_i = \frac{2a_n}{\delta}$ is the transition coefficient and the parameter $\xi\in [\frac12, 1]$ can be used to interpolate
between upwind ($\xi = 1$) and central difference ($\xi=\frac12$) scheme. Equations \eqref{eq:fracture_continuity}, \eqref{eq:fracture_darcy}, and
\eqref{eq:fracture_normal} describe the general form of the advection-diffusion process on the fracture and its communication with 
the surrounding continuum which we shall later apply to individual processes.

The asymptotic problem reads:
\begin{subequations}
\label{eq:asymptotic_pde}
\begin{align}
-\div(\tn A\nabla u_m) &= f &&\mbox{ in }\Omega_m,\\
u_m &= 0 &&\mbox{ on }\partial\Omega\cap\partial\Omega_m,\\
-\tn A\nabla u_m\cdot\vc n_i &= q(u_m,u_f) &&\mbox{ on }\gamma_i,i=1,2,\\
-\div(\tn A\nabla_\yy u_f) &= \bar f + \frac1\delta\sum_{i=1}^2 q(u_{m|\gamma_i},u_f) &&\mbox{ in }\gamma,\\
u_f &= 0 &&\mbox{ on }\overline\gamma\cap\partial\Omega,
\end{align}
\end{subequations}
where $q(w,z):=\frac2\delta a_n(w-z)$ and $\vc n_i$ is the unit outward normal vector to $\Omega_i$, i=1,2, satisfying $\vc n:=\vc n_1=-\vc n_2=(1,0,0)^\top$.




\section{Darcy flow}

We shall apply the formal derivation of the abstract continuum-fracture model to the steady-state Darcian flow.

\subsection{Continuum-fracture model}

Using the notation of the preceding section, we set $u:=h$, $w:=0$, $\tn A:=\tn K$, $f:=f_h$.
We assume that
\[ \tn K = \begin{cases}\tn K_m & \mbox{ in }\Omega_m,\\ \tn K_f = \begin{pmatrix}k_n & 0\\0&\tn K_t\end{pmatrix} & \mbox{ in }\Omega_f,\end{cases} \]
where $k_n(x,\yy)=k_n(\yy)$, $\tn K$ is uniformly positive definite and bounded in $\Omega$ and $f_h\in L^2(\Omega)$.
This results in the new system
\begin{subequations}
\label{eq:asymptotic_darcy}
\begin{align}
-\div(\tn K_m\nabla h_m) &= f_h &&\mbox{ in }\Omega_m,\\
h_m &= 0 &&\mbox{ on }\partial\Omega\cap\partial\Omega_m,\\
-\tn K_m\nabla h_m\cdot\vc n_i &= q(h_m,h_f) &&\mbox{ on }\gamma_i,i=1,2,\\
-\delta\div(\tn K_f\nabla_\yy h_f) &= \delta\bar f_h + \sum_{i=1}^2 q(h_{m|\gamma_i},h_f) &&\mbox{ in }\gamma,\\
h_f &= 0 &&\mbox{ on }\overline\gamma\cap\partial\Omega,
\end{align}
\end{subequations}
where $q(w,z):=\frac2\delta k_n(w-z)$.
\note{JS}{Is it clear what $h_f$ means on $\gamma_i$?}


\subsection{Weak formulation}

The ongoing analysis will be done in the framework of weak solutions.
We say that $h\in H^1_0(\Omega)$ is the weak solution of \eqref{eq:darcy_flow} if for every $v\in H^1_0(\Omega)$:
\begin{equation}
\label{eq:weak_darcy}
\int_\Omega \tn K\nabla h\cdot\nabla v = \int_\Omega f_hv.
\end{equation}
Introducing the space
\[ H^1_{bc}(\Omega_m) := \{v\in H^1(\Omega_m);~v_{|\partial\Omega_m\cap\partial\Omega}=0\}, \]
we analogously define the weak solution of \eqref{eq:asymptotic_darcy} as the couple $(h_m,h_f)\in H^1_{bc}(\Omega_m)\times H^1_0(\gamma)$ that satisfies
\begin{subequations}
\label{eq:weak_asym}
\begin{equation}
\int_{\Omega_m}\tn K\nabla h_m\cdot\nabla v_m + \sum_{i=1}^2\int_\gamma q(h_{m|\gamma_i},h_f)v_{m|\gamma_i} = \int_{\Omega_m} f_hv_m,
\end{equation}
\begin{equation}
\delta\int_{\gamma}\tn K_t\nabla_\yy h_f\cdot\nabla_\yy v_f = \delta\int_\gamma\bar f_h v_f + \sum_{i=1}^2\int_\gamma q(h_{m|\gamma_i},h_f)v_f
\end{equation}
for all $(v_m,v_f)\in H^1_{bc}(\Omega_m)\times H^1_0(\gamma)$.
\end{subequations}
We shall also use a compact form of \eqref{eq:weak_asym}:
\begin{multline}
\int_{\Omega_m}\tn K\nabla h_m\cdot\nabla v_m
+\delta\int_{\gamma}\tn K_t\nabla_\yy h_f\cdot\nabla_\yy v_f\\
+ \sum_{i=1}^2\int_\gamma q(h_{m|\gamma_i},h_f)(v_{m|\gamma_i} - v_f)
 = \int_{\Omega_m} f_hv_m + \delta\int_\gamma\bar f_h v_f.
\end{multline}
Let us remark that under the above assumptions on $\tn K$ and $f_h$, problems \eqref{eq:weak_darcy} and \eqref{eq:weak_asym} have unique solutions.



\subsection{Error analysis of asymptotic model}
\label{sc:error_estimate}

% (note: $h\in H^1(\Omega_f)\Rightarrow\bar h\in H^1(\gamma)$)


We make the following notation:
$\underline K_m := \inf_{\Omega_m,\lambda\in\sigma(\tn K_m(\cdot))}\lambda$, $\underline K_t := \inf_{\Omega_f,\lambda\in\sigma(\tn K_t(\cdot))}\lambda$, $\underline k_n:=\inf_{\gamma}k_n(\cdot)$, $\overline k_n:=\sup_{\gamma}k_n(\cdot)$.
The main result of this section is the following error estimate.
\begin{theorem}
\label{th:error_estimate}
Let $\delta>0$, and assume that the unique solution to \eqref{eq:weak_darcy} satisfies additionally $\partial_n^2 h\in C(\overline\Omega_f)$.
Then there is a constant $C:=C(\Omega,\gamma)>0$ independent of $\delta$, $\tn K$ and $f_h$ such that
\begin{subequations}
\label{eq:error_estimates_delta}
\begin{align}
&\norm{\nabla_\yy(\bar h- h_f)}_{2,\gamma} \le C\sqrt{\frac{\overline k_n}{\underline K_t}}\norm{\partial_n^2 h}_{C(\overline\Omega_f)}\delta,\\
&\norm{\nabla(h-h_m)}_{2,\Omega_m} \le C\sqrt{\frac{\overline k_n}{\underline K_m}}\norm{\partial_n^2 h}_{C(\overline\Omega_f)}\delta^{3/2},\\
&\sum_{i=1}^2\norm{\bar h-h_{|\gamma_i}+h_{m|\gamma_i}-h_f}_{2,\gamma} \le C\sqrt{\frac{\overline k_n}{\underline k_n}}\norm{\partial_n^2 h}_{C(\overline\Omega_f)}\delta^2,
\end{align}
\end{subequations}
where $(h_m,h_f)$ is the solution to \eqref{eq:weak_asym} and $\underline\lambda(\cdot)$ denotes the smallest eigenvalue of a matrix.
\end{theorem}


\begin{proof}
Let us define for any $k\in\Natural$ an auxiliary operator $\Pi_k:L^2(\Omega_m)\times L^2(\gamma)\to L^2(\Omega)$:
\begin{multline*}
\Pi_k(v_m,v_\gamma)(x,\vc y)\\ :=
\begin{cases}
v_m(x,\vc y) & \mbox{ in }\Omega_m,\\
v_\gamma(0,\vc y) & \mbox{ in }\Omega_{fk}:=\{(\tilde x,\tilde{\vc y})\in\Omega;~-\frac\delta2+\frac1k<\tilde x<\frac\delta2-\frac1k\},\\
k(x+\frac\delta2)v_\gamma(0,\vc y) - k(x+\frac\delta2-\frac1k)v_m(-\frac\delta2,\vc y) & \mbox{ in }\{(\tilde x,\tilde{\vc y})\in\Omega;~\tilde x<-\frac\delta2+\frac1k\},\\
-k(x-\frac\delta2)v_\gamma(0,\vc y) + k(x-\frac\delta2+\frac1k)v_m(\frac\delta2,\vc y) & \mbox{ in }\{(\tilde x,\tilde{\vc y})\in\Omega;~\tilde x>\frac\delta2-\frac1k\}.
\end{cases}
\end{multline*}
Note that $\Pi_k$ maps $H^1(\Omega_m)\times H^{1/2}(\gamma)$ into $H^1(\Omega)$.
Hence we can test \eqref{eq:weak_global} by $v_k:=\Pi_k(h-h_m,\bar h-h_f)$, where $h$, $h_m$ and $h_f$ satisfy \eqref{eq:weak_global} and \eqref{eq:weak_asym}, respectively:
\begin{multline}
\label{eq:global_vk}
\int_{\Omega_m}\tn A\nabla h\cdot\nabla(h-h_m)
+\int_{\Omega_{fk}}\tn A_t\nabla_\yy h\cdot\nabla_\yy(\bar h-h_f)
+\int_{\Omega_f\setminus\Omega_{fk}} a_n\partial_n h \partial_n v_k\\
+ \int_{\Omega_f\setminus\Omega_{fk}} \tn A_t\nabla_\yy h \cdot \Pi_k(h-h_m,\nabla_\yy(\bar h-h_f))
= \int_{\Omega_m} f (h-h_m)\\
+ \int_{\Omega_{fk}} f (\bar h-h_f)
+ \int_{\Omega_f\setminus\Omega_{fk}} f v_k.
\end{multline}
Next we shall perform the limit $k\to\infty$.
Due to continuity of integral we have:
\begin{align}
&\int_{\Omega_{fk}}\tn A_t\nabla_\yy h\cdot\nabla_\yy(\bar h-h_f) \to \int_{\Omega_f}\tn A_t\nabla_\yy h\cdot\nabla_\yy(\bar h-h_f)\\
&\hspace{4cm} = \delta\int_\gamma\tn A_t\nabla_\yy\bar h\cdot\nabla_\yy(\bar h-h_f),\\
&\int_{\Omega_f\setminus\Omega_{fk}} \tn A_t\nabla_\yy h \cdot \Pi_k(h-h_m,\nabla_\yy(\bar h-h_f)) \to 0, \\
&\int_{\Omega_{fk}} f (\bar h-h_f) \to \int_{\Omega_{f}} f (\bar h-h_f) = \delta\int_{\gamma} \bar f (\bar h-h_f), \\
&\int_{\Omega_f\setminus\Omega_{fk}} f v_k \to 0,~k\to\infty.
\end{align}
The remaining term can be rewritten as follows:
\begin{multline}
\int_{\Omega_f\setminus\Omega_{fk}} a_n\partial_n h \partial_n v_k
= k\int_{-\frac\delta2}^{-\frac\delta2+\frac1k}\int_\gamma a_n\partial_n h (\bar h - h_f - h_{|\gamma_1} + h_{m|\gamma_1})\\
+ k\int_{\frac\delta2-\frac1k}^{\frac\delta2}\int_\gamma a_n\partial_n h (\bar h - h_f - h_{|\gamma_2} + h_{m|\gamma_2})\\
\to \sum_{i=1}^2\int_\gamma a_n \partial_n h_{|\gamma_i} (\bar h - h_f - h_{|\gamma_i} + h_{m|\gamma_i}),~k\to\infty.
\end{multline}
Since $\partial_n h$ is smooth, Taylor's expansion yields
\[ \partial_n h(\pm\frac\delta2,\vc y) = \frac2\delta(\bar h - h(\pm\frac\delta2,\vc y)) - \frac\delta3\partial_n^2 h(\xi(\vc y)),~\xi(\yy)\in(-\frac\delta2,\frac\delta2), \]
i.e. there exist functions $g_i:=\partial^2_n h(\xi(\cdot))\in L^\infty(\gamma)$ such that
\begin{equation}
\label{eq:taylor_for_du}
\partial_n h_{|\gamma_i} = \frac2\delta(\bar h - h_{|\gamma_i}) - \delta g_i,~i=1,2.
\end{equation}
Summing up, \eqref{eq:global_vk}--\eqref{eq:taylor_for_du} yield:
\begin{multline}
\label{eq:sum_global_vk_limit}
\int_{\Omega_m}\tn A\nabla h\cdot\nabla(h-h_m)
+\delta\int_\gamma\tn A_t\nabla_\yy\bar h\cdot\nabla_\yy(\bar h-h_f)\\
+ \sum_{i=1}^2\int_\gamma q(\bar h,h_{|\gamma_i}) (\bar h - h_{|\gamma_i} + h_{m|\gamma_i} - h_f)\\
= \int_{\Omega_m} f (h-h_m)
+ \delta\int_{\gamma} \bar f (\bar h-h_f)
+ \delta\sum_{i=1}^2\int_\gamma a_n g_i\, (\bar h - h_{|\gamma_i} + h_{m|\gamma_i} - h_f).
\end{multline}

Now we plug into \eqref{eq:weak_asym} test functions $h-h_m$, $\delta(\bar h-h_f)$, respectively, and add the resulting equations.
We obtain:
\begin{multline*}
\int_{\Omega_m}\tn A\nabla h_m\cdot\nabla(h-h_m) + \delta\int_\gamma\tn A_t\nabla_\yy h_f\cdot\nabla_\yy(\bar h-h_f)\\
- \sum_{i=1}^2\int_\gamma q(h_{m|\gamma_i},h_f)(\bar h - h_{|\gamma_i}+h_{m|\gamma_i} - h_f)
= \int_{\Omega_m} f(h-h_m)
+ \delta\int_\gamma\bar f(\bar h-h_f).
\end{multline*}
Subtracting this from \eqref{eq:sum_global_vk_limit}, using H\"older's and Young's inequality yields:
\begin{multline}
\int_{\Omega_m}\tn A\nabla (h-h_m)\cdot\nabla(h-h_m)
+\delta\int_\gamma\tn A_t\nabla_\yy(\bar h-h_f)\cdot\nabla_\yy(\bar h-h_f)\\
+ \sum_{i=1}^2\int_\gamma \frac{2a_n}\delta |\bar h - h_{|\gamma_i} + h_{m|\gamma_i} - h_f|^2
= \delta\sum_{i=1}^2\int_\gamma a_n g_i\, (\bar h - h_{|\gamma_i} + h_{m|\gamma_i} - h_f)\\
\le \frac{\delta^{\frac32}}{\sqrt2}\sum_{i=1}^2\int_\gamma \sqrt{a_n}|g_i|\sqrt{\frac{2a_n}\delta}|\bar h - h_{|\gamma_i} + h_{m|\gamma_i} - h_f|\\
\le \frac{\delta^3}4\norm{a_n}_{\infty,\gamma}\sum_{i=1}^2\norm{g_i}_{2,\gamma}^2 + \frac12\sum_{i=1}^2\int_\gamma \frac{2a_n}\delta |\bar h - h_{|\gamma_i} + h_{m|\gamma_i} - h_f|^2.
\end{multline}
Finally we absorb the last term in the left hand side and use uniform positive definiteness of $\tn A$:
\begin{multline}
\underline\lambda(\tn A_{|\Omega_m})\norm{\nabla (h-h_m)}_{2,\Omega_m}^2
+\underline\lambda(\tn A_t)\delta\norm{\nabla_\yy(\bar h-h_f)}_{2,\gamma}^2\\
+ \frac1\delta\inf_{\Omega_f}a_n\sum_{i=1}^2\norm{\bar h - h_{|\gamma_i} + h_{m|\gamma_i} - h_f}_{2,\gamma}^2
\le \frac{\norm{a_n}_{\infty,\gamma}}4\left(\sum_{i=1}^2\norm{g_i}_{2,\gamma}^2\right)\delta^3,
\end{multline}
from which the estimates \eqref{eq:error_estimates_delta} follow.
\end{proof}

\section{Solute transport}
\subsection{Continuum-fracture model}
\subsection{Weak formulation}
\subsection{Error analysis of asymptotic model}


\section{Numerical experiments}
\label{sc:numerics}

\note{JS}{
OUR STRATEGY - compare two models using 2nd order approximation in space + meshes refined proportionally to cross section.
\\
DESCRIBE FLOW123D - mixed-hybrid method for Darcy flow, DG for transport, input/output interface
\\
TEST PROBLEMS - analytical solutions, tests from JMR paper, comments to the results
}














%\input{heat}



\bibliographystyle{abbrvnat}
\bibliography{flow123d_doc.bib}
%%%%%%%%%%%%%%%%%%%%%%%%%%%%%%%%%%%%%%%%%%%%%%%%%%%%%%%%%%%%%%%%%%%%%%%%%%%%%%%%%%%%%%%%%%%%%%%%%%%%%%%%%%%%%%%%%


\end{document}


